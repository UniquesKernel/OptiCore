\section{ Introduction }
\label{sec:Introduction}

\simulator{} is a virtualized computational platform designed to abstract
and explore the core principles of CPU architecture and program optimization.
By eliminating the complexities of physical hardware, the platform provides
an accessible environment for analyzing CPU behavior, assembly-level programming,
and optimization techniques

\vspace{1em}

\noindent The \simulator{} platform consist of the following principal
components:

\begin{enumerate}
  \item \textbf{A minimal CPU architecture $\mathcal{A}$:} A custom, minimalistic architecture $\mathcal{A} = (R, M, I, S)$
    where $R$ denotes a register set, $M$ denotes a memory space, $I$ denotes an instruction 
    set, and $S$ denotes a state transition function. The architecture is intentionally 
    constrained to encapsulate the core features of a CPU while abstracting away non-essential
    complexity. 
  \item \textbf{Assembler $\mathcal{T} : L \rightarrow B$}: A deterministic mapping $\mathcal{T}$ from the 
    assembly language $L$ to a binary, but equivalent, representation $B$. where $\forall P \in L, \mathcal{T}(P) = P_{b}$ 
    and $P_{b} \in B$ is a sequence of binary instructions. This ensures an exact translation of assembly-level
    instructions to machine code compatible with the architecture $\mathcal{A}$.
  \item \textbf{Simulator S:} A simulator function $S$, which, when applied to a binary program $P_{b} \in B$,
    simulates the execution of of the instruction in $P_{b}$ according to the Architecture $\mathcal{A}$. The simulator $S$ defines 
    the sequence of state transitions $\{S_i\}_{i = 0}^{n}$ corresponding to the instruction sequence $P_b$, where each
    state $S_k \in S$ is a valid configuration of the registers, memory and program counter.
  \item \textbf{Optimization Tool $\mathcal{O}(P_{b})$:} A tool $\mathcal{O}$ that applies a 
    function $\mathcal{O} : B \rightarrow B'$, where $B'$ is a binary program that optimizes $P_{b}$ by reducing
    the execution time, or resource consumption. The optimization respects the constraints imposed 
    by the architecture $\mathcal{A}$ and preserves the semantic equivalence of $P_{b}$ and 
    $B'$, i.e. $S(P_{b}) = S(B')$ in terms of output state.
\end{enumerate}

\subsection{Custom Assembly and Program Execution}
\label{subsec:Custom-Assembly}

Users interact with the \simulator{} platform through a custom-designed assembly language $L$.
Programs $P \in L$ are composed of a sequence of instructions $\{I_i\}_{i = 0}^{n}$, where each 
instruction, $I_{k}$ conforms to the specifcations of $\mathcal{A}$'s instruction set $I$. The 
assembler $T$ translate $P$ into its binary Equivalent $P_{b} \in B$, which is subsequently 
executed by the simulator $S$.

the simulator mimics the behavior of the minimal CPU architecture $\mathcal{A}$, providing the 
emulation of instruction execution, register manipulation, memory access, and state transtions. 
Formally, the program $P_{b}$ produces a sequence of states $\{S_i\}_{i=0}^n$, where $S_k = A(S_{k - 1}, I_{k})$,
and $\mathcal{A}(S,I)$ denotes the state transtions function applied to the current state $S$ under 
the instruction $I$.

\subsection{Experimental Feature: Optimization Tool}
\label{subsec:Experimental-Tool}

An experimental feature of the \simulator{} platform is the inclusion of an optimization 
tool, denoted $\mathcal{O}$. This tool operates in conjuction with the Simulator to analyze the 
current state of the program $P_{b}$ amd apply optimizations dynamically during the 
the execution of $P_{b}$. Let $S(t)$ denote the state of the system at time $t$ and let 
$P_{b}(t)$ represent the portion of the program being executed at time $t$. The optimization 
function $\mathcal{O}$ attempt to compute and optimize the program state 
$P'_{b}(t) = \mathcal{O}(P_{b}(t), S(t))$, such that:

\begin{equation}
  \label{eq:}
  S(P_{b}) = S(P'_{b}),
\end{equation}

Where $P'_{b}$ is semantically equivalent to $P_{b}$, but potentially more efficient in terms 
of execution time or resource consumption. this process aims to emulate the optimization 
procecss found in strategies such as dynamic code patching, though it is applied within the 
virtual environmento that is the \simulator{} platform.

The effectiveness and development scope of $\mathcal{O}$ are constrained by two factors; 
the formal limitations imposed by the architecture $\mathcal{A}$ and the state of development 
of the \simulator{} platform. If the state of \simulator{} prevents its development, then the 
development of $\mathcal{O}$ becomes either partial or remains in the realm of theoretical 
considerations.

\newpage
